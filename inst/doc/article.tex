%%%%%%%%%%%%%%%%%%%%%%%%%%%%%%%%%%%%%%%%%%%%%%%%%%%%%%%%%%%%%%%%%%%%%%
% Danny Arends
% The qcl mapping routine
%\VignetteIndexEntry{Article}
%
%%%%%%%%%%%%%%%%%%%%%%%%%%%%%%%%%%%%%%%%%%%%%%%%%%%%%%%%%%%%%%%%%%%%%%
\documentclass[12pt]{article}

\usepackage{scicite}
\usepackage{times}
\usepackage{color}
\usepackage{Sweave}

% revise margins
\topmargin 0.0cm
\oddsidemargin 0.2cm
\textwidth 16cm 
\textheight 21cm
\footskip 1.0cm

%The next command sets up an environment for the abstract to your paper.

\newenvironment{sciabstract}{%
\begin{quote} \bf}
{\end{quote}}


% If your reference list includes text notes as well as references,
% include the following line; otherwise, comment it out.

\renewcommand\refname{References and Notes}

% The following lines set up an environment for the last note in the
% reference list, which commonly includes acknowledgments of funding,
% help, etc.  It's intended for users of BibTeX or the {thebibliography}
% environment.  Users who are hand-coding their references at the end
% using a list environment such as {enumerate} can simply add another
% item at the end, and it will be numbered automatically.

\newcounter{lastnote}
\newenvironment{scilastnote}{%
\setcounter{lastnote}{\value{enumiv}}%
\addtocounter{lastnote}{+1}%
\begin{list}%
{\arabic{lastnote}.}
{\setlength{\leftmargin}{.22in}}
{\setlength{\labelsep}{.5em}}}
{\end{list}}


\title{QCL: Added value of differential correlation} 


% Place the author information here.  Please hand-code the contact
% information and notecalls; do *not* use \footnote commands.  Let the
% author contact information appear immediately below the author names
% as shown.  We would also prefer that you don't change the type-size
% settings shown here.

\author
{Danny Arends,$^{1,2,\ast}$ Ritsert C. Jansen,$^{1}$\\
\\
\normalsize{$^{1}$Groningen Bioinformatics Centre - University of Groningen}\\
\normalsize{Nijenborgh 7, 9747 AG Groningen, The Netherlands}\\
\normalsize{$^{2}$Genomics Coordination Centre - University Medical Centre Groningen}\\
\normalsize{Hanzeplein 1, 9700 RB Groningen, The Netherlands}\\
\\
\normalsize{$^\ast$To whom correspondence should be addressed; E-mail:  Danny.Arends@Gmail.com.}
}

% Include the date command, but leave its argument blank.

\date{}

%%%%%%%%%%%%%%%%% END OF PREAMBLE %%%%%%%%%%%%%%%%

\begin{document}

% Double-space the manuscript.

\baselineskip24pt

% Make the title.

\maketitle 


\setkeys{Gin}{width=\textwidth} %% <- change width of figures
% Try to get the R code from running into the margin


% Change S input/output font size
\DefineVerbatimEnvironment{Sinput}{Verbatim}{fontsize=\footnotesize, baselinestretch=0.75, formatcom = {\color[rgb]{0, 0, 0.56}}}
\DefineVerbatimEnvironment{Soutput}{Verbatim}{fontsize=\footnotesize, baselinestretch=0.75, formatcom = {\color[rgb]{0.56, 0, 0}}}

\begin{sciabstract}
  In this paper we present Quanitative Correlation Locus (QCL) mapping, a novel approach 
  to detect genetic regulation of phenotypes in recombinant inbred line populations (RIL) 
  or SNP genotypes populations. QCL mapping allows association of phenotype correlation 
  differences to genetic markers, in almost the same way as QTL mapping associates genetic 
  markers with differences in expression.
  We use QCL mapping to reconstruct underlying genetic architecture, and provide insight 
  into regulation of (gene) expression. QCL information in combination with classical QTL 
  information leads to a more detailed insight into the genetic architecture underlying 
  complex traits. The method also provides a missing link in regulatory network reconstruction.
\end{sciabstract}

\section*{Introduction}

  Genetical genomics experiments have shown that gene expression variation can be mapped to 
  genotypic variation \cite{Jansen:2001}. Quantitative Trait Locus (QTL) mapping of a 
  gene identifies regions in the genome for which different genotypes lead to differential 
  expression level of the gene. This QTL mapping can be done on all molecular levels including, 
  gene expression (eQTL), protein abundance (pQTL) and metabolites (mQTL).

  Recently different methods have been proposed that allow studying, in addition to the genes 
  differentially expressed between conditions, the genes that are differentially coexpressed 
  between conditions \cite{Tesson:2010,Kompass:2011}. Such differential co-expression 
  is taken as another level of information from genomic data. This level of information can for 
  example be used to expose transcriptional network rewiring \cite{Langfelder:2008}. Differential 
  co-expression has been observed in studies of different diseases, including between healthy and 
  cancer tissue studies. In parallel, correlations between phenotypes are studied for various 
  reasons e.g. discovering and detecting batch effects, marker assisted breeding and causal 
  inference. The importance of identifying differences in correlation has been increasingly 
  recognized and this led to the development of this algorithm to perform QCL analysis in 
  experimental crosses.

  Using recombinant inbred lines we introduce natural variation into a population 
  detectable as cis-QTL in a genome wide QTL scan. The effect of any such genetic 
  variation is transfered downstream to interaction partners. When this leads to 
  changes in mean expression of one of the partners we can also detect this using 
  classical QTL methods (trans/distant QTL). 

  However QTL mapping suffers from two major drawbacks (p1) the interaction partners 
  of genes are obscured by many phenotypes co-localizing at a single genetic marker, and 
  (p2) due to buffering the effect of any variation will 'fade' quickly and not be 
  detectable as the difference in mean phenotype in the population does not change 
  significantly.

  This inspired us to re-think parts of the basic assumptions in genetics:
  Imagine phenotype A shows a QTL, we suppose that near this marker something (e.g. DNA variation, 
  methylation, promotor sequences, etc) introduces expression variation in A. If we now assume there is a 
  phenotype B (downstream to A) variation of expression in phenotype A is causing 'variation' 
  for phenotype B. However B will in most cases not show variation because it's expression 
  depends (but is not dependant) on A, a phenomena generally refered to as phenotype buffering. 
  This is what QCL mapping exploits, the hypothesis is that although B doesn't show a QTL, 
  it will become uncorrelated when we partition phenotype B into the groups based on the genotype at 
  the QTL marker of phenotype A. 

  QCL thus maps not differential co-expression but differential correlation to genetic 
  variation, i.e. to identify regions in the genome for which one genotype value leads to 
  correlated expression between a phenotype and all other phenotypes, exploiting the fact 
  that any QTL will be the result of either direct induced variation or indirect variation, 
  which will be visible as correlation differences among individuals carrying different genotypes. 
  We merge information coming from QCL and QTL to reveal underlying genetic architecture, 
  which would remain hidden when using only classical QTL information.

  We urge all QTL researchers to combine the new QCL method with existing QTL analysis to expose 
  transcriptional networks involved in the regulation of complex phenotypes. For this we 
  provide an R package with a reference implementation of the QCL algorithm.
\bigskip 

\section*{Methodology}
\nopagebreak

\emph{\sffamily QCL mapping}\\
  As mentioned in the introduction QCL mapping uses buffering to detect interaction partners. 
  When we split the population based on genotype at a QTL marker, only related phenotypes 
  will loosing correlation with the gene under investigation. This allows us to not only 
  only identify its interaction partners, but also means QCL mapping does not suffering from 
  (p1) many phenotypes co-localizing. Additionally it is more sensitive to detect variation 
  caused by regulatory phenotypes because variation of these regulators causes many 'downstream' 
  phenotypes to show differential correlation with the inducing phenotype, because QCL relies on 
  correlation changes as opposed to differences in phenotype mean QCL is unaffected by (p2).

\emph{\sffamily Choosing a start/seed probe (Optional)}\\
  Choose a phenotype to subject to QCL analysis, this can be your favourite gene, 
  protein, metabolite or classical phenotype.

\emph{\sffamily Single phenotype QCL mapping}\\
  Using genotype data we can split individuals at each marker in two groups (A or B).
  At each marker calculate the correlation between our phenotype and all other 
  phenotypes in groups A and B. Collect all the result vectors into two matrices A 
  and B, and calculate the QCL matrix by taking the squared difference between matrix A 
  and B.

\emph{\sffamily Significance testing}\\
  To compare to QTL mapping we need to convert the QCLscores to LOD scores. Significance 
  of QCL were assessed by using permutations, in each round the link between genotype and 
  phenotype is broken, by redistributing at random genotypes amongst individuals not 
  allowing for duplicates. The QCL on these new individuals allows for estimation of 
  the underlying null-distribution of QCL scores. We frequently observe QCL scores in 
  real data which is higher than any QCL score obtained during permutation. When this 
  happens we use a generalized Pareto distribution (GPD) to estimate the extreme tail 
  of the distribution \cite{Knijnenburg:2009}, to also get likelyhood estimates for the 
  extreme scores observed. See figures XX, and XX showing the score distribution in 
  permuted data, Fig XX also shows a tail estimation using the GPD method.

\emph{\sffamily Creating single trait QCL profiles}\\
  After permutation single trait QCL profiles can be obtained by transforming the QCL scores 
  into LOD scores. For this we use a look-up table of scores obtained during permutation, we 
  assess the likelyhood and take the $-log10(P.value)$ of every QCL score see Fig. 
  \ref{fig:singletraitLOD}. This is done 1) to compare QTL profiles with summarized QCL profiles 
  (an example in Fig. \ref{fig:singletrait}), and 2) Show the contribution of each phenotype 
  to the total likelyhood.

\emph{\sffamily Visualizing the Phenotypes x Phenotypes QCL relations}\\
  In Fig. \ref{fig:singletraitLOD} we see the additional information obtained by QCL mapping summarized 
  graphically.  We can also calculate the full Phenotypes x Phenotypes matrix. This shows the summed 
  evidence of all paiwise interaction between multiple phenotypes.

\emph{\sffamily Visualizing the Phenotypes x Marker relations}\\
  When plotting multiple QCL profiles as a heatmap, QCLs are seen mapping to the 
  same genetic location. At these locations, regulation is taking place. We expect 
  to find one (or more) of the genes to show a cis-QTL, this is because a QCL can 
  only exist if there is a QTL inducing it.

\emph{\sffamily Network reconstruction}\\
  So how do we build up gene interaction networks, This looks complicated but the information 
  is implicitely stored in the Phenotype x Marker and Phenotype x Phentype relations matrices. 
  By using network scanning we only identify part of the network in which our phenotype is 
  active and closely related interaction partners. To identify a pathway or main regulatory phenotypes 
  controlling our phenotype we need to combine this information.

  We show two examples in which we reconstruct the genetic architecture using QCL mapping. 
  Both datasets are available inside the R package, as well as the code to analyse these 
  two datasets.

\subsection*{Materials}
\emph{\sffamily Dataset Example 1. A. Thaliana}\\
  Metabolite data from an A. Thaliana Landsberg erecta (Ler) and Cape Verde Islands (Cvi) 
  dataset from R/qtl, 24 characterized metabolites which are the products of chain elongation. 
  There are 162	recombinant inbred lines genotyped at 117 markers on 5 chromosomes. Additional 
  details about the experimental settings are available in the original paper \cite{Keurentjes:2006,AlonsoBlanco:1998}

\emph{\sffamily Dataset Example 2. S. Cerevisiae}\\
  S. Cerevisiae dataset from the gene expression omnibus (GEO) \cite{Edgar:2002} (GEO 
  accession GDS1115, and GDS1116, \cite{Brem:2005}. This dataset contains expression 
  profiles of parental strains and haploid progenies from a cross of strain BY4716 and 
  the wild wine strain RM11-1a. Data was collected for 4482 probes. Genotype data 
  for the 109 progenies was available at 282 genetic markers obtained by contacting 
  the authors of the original paper. 
  Expression of probes was measured under growth conditions containing 1\% ethanol. 
  Additional details about the experimental settings are available in the original 
  paper \cite{Brem:2005}.

\bigskip
\section*{Results}
\emph{\sffamily Preprocessing}

  Probes showing little to no variance (< 0.01) were removed from the analysis. Batch 
  effects due to different hybridization dates were estimated and removed from the data 
  by using a mean centering approach (Group means were estimated and corrected).

\nopagebreak

  \subsection{Example 1. A. Thaliana}
  QCL mapping was performed on 24 characterized metabolities measured in Arabidopsis 
  Thaliana. A significance threshold for QCL mapping was determined for each trait 
  by using 1000 permutations, the outcomes can be seen in Fig \ref{fig:histPermAra}. 
  We used a very stringent QCL threshold at P < 0.001 to use as 'significant' QCL 
  score for further analysis. 
  
  \emph{\sffamily Significance}\\
  The major observations is that we cannot use a 'flat' cut-off for all traits, each 
  trait shows a distinct NULL-distribution. A squared correlation difference of 0.5 is 
  significant for X3.Hydroxypropyl (green). But for X4.Methylsulfinylbutyl we need a 
  difference in squared correlation larger then 0.7 (yellow).
  
  \emph{\sffamily Single trait QCL}\\
  The 24 QCL profiles were compared to the QTL profiles previously determined. We observed 3 cases:
  1) Additional QCL found, where no QTL was present as seen in the QCL profile X3.Hydroxypropyl (Fig. \ref{fig:singletraitLOD}), 
  These are the cases in which the loss of variation in the genes underneath the QCL is not causing 
  variation differences in the observed phenotype, but correlation differences.
  2) QCL co-localized with QTL, (Also QCL gives a detailed view of which phenotypes are underneath its peek, we 
  interpret this as variation transfered from the phenotypes underneath the QCL to the phenotype investigated.
  3) QTL with no QCL, we interpret this as the source of the variation.
  Analysis in this way of all 24 Single trait QCL profiles combined with their QTL profiles 
  allows us to reconstruct the source of natural variation and follow it down the network.
  
  \emph{\sffamily QCL combined with QTL}\\
  Plots of the results of the QCL mapping in Arabidopsis can be found in 
  Figure XX Phenotype to Marker interaction, which is comparable to the QTL mapping 
  (Fig. XX). The additional information obtained by QCL mapping is shown in a heatmap 
  of the Phenotype to Phenotype interaction matrix (figure XX) and the same information 
  but now represented as an interaction network is seen in figure XX and in the QCL plot
  
  \emph{\sffamily Reconstructed network}\\
  Using the derived data we can begin to reconstruct the network. We'll start with a metabolite 
  which shows mappable variation changes, but no QCL colocatizing (Inducer).
  
  \subsection{Example 2. S. Cerevisiae}
  QCL mapping is computationaly expensive (We need 1000+ permutations for each phenotype), 
  and for the S. Cerevisiae we pre-selected a background of 300 probes which showed the 
  highest variation, and had been previously been identified as having QTLs. Significance 
  thresholds for QCL mapping were determined for each trait by using 1000 permutations 
  ({\emph Data not shown}).
  
  \emph{\sffamily Single trait QCL}\\
  

\bigskip
\section*{Discussion}
\nopagebreak

  Microarray experiments have identified genes differentially expressed between various 
  conditions, or groups of genes coexpressed across samples. Genetical genomics screens 
  have been carried out to map differential expression to genetic variation. Screening 
  the genome for genotype combinations for which there is differential expression has 
  been a successful strategy when mapping Mendelian traits. 
  
  However results on complex traits have been less then satisfactory with only small 
  amounts of heritability explained by individuals loci. QCL mapping however 'borrows' 
  information from each phenotype in the analysis set, and enables extraction of 
  phenotype to marker to phenotype networks, the flow of variation in the network. 
  Furthermore QCLs seem to 'transfer' from phenotype to phenotypes in a stabile 
  fashion, which gives greater detail and confidence when reconstructing genetic 
  architecture. Also QCL allows us to better identify gene transcript introducing 
  variation, by their absense under QTL peeks which introduce variation, compared 
  to being present underneath QTL which are affected by upstream variation.
  
  It provides better insight in the missing heritability observed in phenotypes by 
  allowing it to be bufferen and thus be hiden for classical QTL mapping approaches, we 
  hypothesize that changes in variation are affecting (by loss of) correlation in 
  the normally correlated phenotypes leading to an amplification of this variation, 
  by distrupting the network environment of which endo-phenotypes are part of.

\section*{Acknowledgements}
  This work was supported by the Centre for BioSystems Genomics (CBSG) and the 
  Netherlands Consortium of Systems Biology (NCSB), both of which are part of the Netherlands 
  Genomics Initiative / Netherlands Organisation for Scientific Research [to DA], and 
  by the EU 7th Framework Programme under the Research Project PANACEA, [222936 to RJ].

Also we'd like to mention Joeri van der Velde and Konrad Zych for supporting me with some very 
valuable feedback on figures presented here.

We'd like to thank readers for their attention, and invite them to 
address any questions to Danny Arends, at Danny.Arends@gmail.com.

\newpage
\bibliography{scibib}
\bibliographystyle{Science}

\newpage

\section*{Figures}

\begin{figure}[ht]
  \centerline{\includegraphics{QCL.pdf}}
  \caption{QCL mapping: The underlying hypothesis of QCL mapping. Seen here is a network of 3 phenotypes, in 
  which expression of P1, P2 and P3 is measured, and we want to infer the hidden relation between P1 
  and P2. In our model a genetic pertubation is causing P1 to change in expression showing a QTL at a marker. We expect variation 
  induced by P1 to transfer to P2, However (like in most biological systems) expression of P2 is tightly controlled, 
  and variation seen in phenotype 1 wil not induce a QTL in phenotype 2. However when we partition individuals based 
  on the genotype at the QTL marker of P1, we can make use of the correlation structure between P1 and P2. Allowing 
  us to infer the relationship between phenotype 1 and 2, when we see difference in correlation between P1A and P2A 
  and P1B and P2B. We don't observe this difference in correlation between P3 and P1, meaning there is no relation 
  between P1 and P3}
  \label{QCL_img}
\end{figure}

\begin{figure}[ht]
  \centering
\includegraphics{fig-permutation_scores}
  \caption{QCL scores obtained during permutations (A. Thaliana) additional information obtained by QCL mapping, 
  which can be used to reconstruct the underlying genetic architecture.\label{fig:histPermAra}}
\end{figure}

\begin{figure}
  \centering
\includegraphics{fig-singletrait}
  \caption{QCL profile (A. Thaliana, X3.Hydroxypropyl) transformed to LOD scores compared to the QTL profile of the same phenotype, We see an 
  additional significant peek for this phenotype (Between Marker 50 to 60) \label{fig:singletrait}}
\end{figure}

\begin{figure}
  \centering
\includegraphics{fig-singletraitLOD}
  \caption{QCL profile (A. Thaliana, X3.Hydroxypropyl) transformed to LOD scores. Shown is the likelyhood contribution of each of 
  the 24 metabolites to the QCL profile. \label{fig:singletraitLOD}}
\end{figure}

\begin{figure}[ht]
  \centering
\includegraphics{fig-p2mmatrix}
  \caption{Phenotypes x Markers matrix (A. Thaliana) this is 'comparable' to a QTL scan on multiple traits.\label{fig:p2mmatrix}}
\end{figure}

\begin{figure}[ht]
  \centering
\includegraphics{fig-QTLmatrix}
  \caption{Heatmap showing QTL scores for all 24 metabolities in A. Thaliana.\label{fig:QTLmatrix}}
\end{figure}

\begin{figure}[ht]
  \centering
\includegraphics{fig-p2pmatrix}
  \caption{Phenotypes x Phenotypes matrix (A. Thaliana) additional information obtained by QCL mapping, 
  which can be used to reconstruct the underlying genetic architecture.\label{fig:p2pmatrix}}
\end{figure}

\begin{figure}[ht]
  \centerline{\includegraphics{AraNet.pdf}}
  \caption{Reconstructed network for chain elongation in Arabidopsis Thaliana}
  \label{AraNet}
\end{figure}

\end{document}
